\documentclass[11pt]{article}
\usepackage[margin=1in]{geometry}
\usepackage{graphicx}
\usepackage{booktabs}
\usepackage{parskip}
\usepackage{hyperref}

\title{HG Pre-Expiry (DTE) Analysis \\ 2008--2024}
\author{Futures Curve Pipeline (Stages 0--2)}
\date{\today}

\begin{document}
\maketitle

\section*{What this report contains (analytics only)}
This document summarizes the \textbf{analytic outcomes} of a DTE-anchored (business-day) pre-expiry study for HG calendar spreads, based on daily US-session VWAP series derived from the Stage 2 spread panel. Reproducibility commands and technical implementation details live in \texttt{README.md}.

\section*{Key findings (headline)}
\begin{itemize}
  \item The strongest in-sample family (2008--2024) in a constrained DTE window search is typically \textbf{short S1} with an \textbf{entry-day S2 regime filter} (``S2$\ge$0 at entry'').
  \item A representative fixed rule selected on 2008--2024 is: \textbf{Short S1, enter at DTE=20, exit at DTE=0, only if S2(entry)$\ge$0}, with round-trip costs included (1 tick/leg/side).
  \item The fixed-rule robustness check shows that this rule is \textbf{sensitive to execution timing}: using \textbf{shift=1} (next-business-day execution) flips performance negative in the pooled sample.
\end{itemize}

\section*{Definitions and conventions (minimal)}
\textbf{Spreads.} S1 = F2 -- F1, S2 = F3 -- F2.

\textbf{DTE.} The x-axis is \textbf{business days to expiry} (DTE), aligned to the near leg of the spread.

\textbf{Daily series.} Baseline price proxy is \textbf{US-session VWAP} (buckets 1--7) with \textbf{shift=0} (same-day execution convention).

\textbf{Costs.} All net metrics include a round-trip cost model of \textbf{1 tick per leg per side} (2 legs, entry+exit).

\section*{DTE-anchored seasonality (S1)}
The seasonality view aligns each contract cycle by DTE and summarizes the cross-cycle distribution (mean + IQR).

\subsection*{Overall average (2008--2024)}
\begin{center}
\includegraphics[width=0.95\textwidth]{output/expiry_seasonality/s1_level_2008_2024_us_vwap_shift0_overall_average.png}
\end{center}
\begin{center}
\includegraphics[width=0.95\textwidth]{output/expiry_seasonality/s1_diff_2008_2024_us_vwap_shift0_overall_average.png}
\end{center}

\subsection*{Why this matters for strategy search}
The overall DTE-anchored curves guide which DTE subwindows might plausibly contain a consistent drift and help motivate a constrained grid search over entry/exit windows.

\section*{Regime-stratified event study (entry-day S2 regime)}
We split contract cycles into regimes using the \textbf{entry-day} S2 sign at a chosen entry DTE (here, DTE=20). The plots show the mean path and dispersion conditional on that entry regime.

\subsection*{Cumulative drift: $S1 - S1@\mathrm{DTE}=20$}
\begin{center}
\includegraphics[width=0.95\textwidth]{output/event_study/s1_us_vwap_shift0__s2_us_vwap_shift0__entry20/cumulative_s1/cumulative_s1_from_dte20_s2_pos.png}
\end{center}
\begin{center}
\includegraphics[width=0.95\textwidth]{output/event_study/s1_us_vwap_shift0__s2_us_vwap_shift0__entry20/cumulative_s1/cumulative_s1_from_dte20_s2_neg.png}
\end{center}

\subsection*{Daily drift: $\Delta S1$ by DTE}
\begin{center}
\includegraphics[width=0.95\textwidth]{output/event_study/s1_us_vwap_shift0__s2_us_vwap_shift0__entry20/dte_drift_ds1/dte_drift_ds1_s2_pos.png}
\end{center}

\section*{Window scan (costed; constrained search)}
We scan DTE windows with entry DTE in [20..5] and exit DTE in [4..0], for both long and short S1. Metrics are computed per contract cycle; net P\&L subtracts the round-trip spread cost.

\subsection*{Heatmaps (S2$\ge$0 regime; short S1)}
\begin{center}
\includegraphics[width=0.95\textwidth]{output/strategy_scan/heatmap_s2_pos_short_net_mean.png}
\end{center}
\begin{center}
\includegraphics[width=0.95\textwidth]{output/strategy_scan/heatmap_s2_pos_short_net_t.png}
\end{center}

\section*{Validation (walk-forward, out-of-sample)}
We run an expanding-window walk-forward procedure by expiry year (train on cycles with expiry year $<Y$, test on cycles with expiry year $=Y$). Selection is constrained with minimum trade counts and a positive net mean requirement, and evaluated out-of-sample per year.

\section*{Robustness (fixed rule; execution/cost stress)}
We take a single baseline rule (selected in-sample) and re-evaluate it under alternative execution assumptions (shift=1, bucket1 proxy, doubled cost). This isolates sensitivity to execution timing and cost assumptions without re-optimizing parameters.

\clearpage
\subsection*{Top ranked windows (in-sample; costed)}

\footnotesize

\begin{tabular}{lrrrrrrr}
\toprule
Regime & Dir & Entry & Exit & N & Net mean (USD/ct) & Net Sharpe-like & Net t \\
\midrule

S2 $\geq$ 0 & Short & 18 & 4 & 122 & 54.29 & 0.29 & 3.17 \\
S2 $\geq$ 0 & Short & 18 & 1 & 122 & 59.72 & 0.26 & 2.89 \\
S2 $\geq$ 0 & Short & 18 & 2 & 121 & 43.61 & 0.26 & 2.81 \\
S2 $\geq$ 0 & Short & 18 & 3 & 122 & 42.27 & 0.22 & 2.44 \\
S2 $\geq$ 0 & Short & 18 & 0 & 121 & 38.19 & 0.21 & 2.33 \\
S2 $\geq$ 0 & Short & 19 & 2 & 122 & 41.29 & 0.20 & 2.16 \\
S2 $\geq$ 0 & Short & 17 & 1 & 117 & 33.23 & 0.19 & 2.09 \\
S2 $\geq$ 0 & Short & 19 & 4 & 122 & 39.15 & 0.19 & 2.11 \\
S2 $\geq$ 0 & Short & 20 & 1 & 109 & 54.35 & 0.19 & 1.95 \\
S2 $\geq$ 0 & Short & 17 & 4 & 117 & 32.14 & 0.19 & 2.01 \\

\bottomrule
\end{tabular}

\normalsize

\subsection*{Walk-forward out-of-sample summary}

\footnotesize

\begin{tabular}{lrrrrrrrrr}
\toprule
Year & Regime & Dir & Entry & Exit & N & Net mean (USD/ct) & Net Sharpe-like & Net t & Win \\
\midrule

2013 & All & Short & 12 & 0 & 12 & -40.06 & -0.40 & -1.38 & 33.3\% \\
2014 & All & Short & 12 & 0 & 12 & -201.20 & -0.52 & -1.80 & 33.3\% \\
2015 & S2 $\geq$ 0 & Short & 12 & 2 & 5 & -92.75 & -0.31 & -0.69 & 40.0\% \\
2016 & S2 $\geq$ 0 & Short & 17 & 0 & 9 & -25.04 & -0.34 & -1.03 & 33.3\% \\
2017 & S2 $\geq$ 0 & Short & 18 & 2 & 9 & 27.69 & 0.30 & 0.89 & 66.7\% \\
2018 & S2 $\geq$ 0 & Short & 17 & 0 & 10 & 68.15 & 0.77 & 2.45 & 90.0\% \\
2019 & S2 $\geq$ 0 & Short & 18 & 2 & 8 & 5.20 & 0.08 & 0.24 & 37.5\% \\
2020 & S2 $\geq$ 0 & Short & 18 & 2 & 6 & -40.79 & -0.40 & -0.97 & 33.3\% \\
2021 & S2 $\geq$ 0 & Short & 18 & 2 & 7 & -152.01 & -1.39 & -3.67 & 0.0\% \\
2022 & S2 $\geq$ 0 & Short & 18 & 4 & 5 & 188.10 & 0.69 & 1.54 & 80.0\% \\
2023 & S2 $\geq$ 0 & Short & 18 & 4 & 9 & 46.02 & 0.39 & 1.18 & 55.6\% \\
2024 & S2 $\geq$ 0 & Short & 18 & 4 & 7 & 74.01 & 0.62 & 1.63 & 57.1\% \\
Pooled &  &  &  &  & 99 & -20.69 & -0.10 & -1.03 & 46.5\% \\

\bottomrule
\end{tabular}

\normalsize

\subsection*{Fixed-rule robustness (pooled)}

\footnotesize

\begin{tabular}{lrrrr}
\toprule
Scenario & N & Net mean (USD/ct) & Net t & Win \\
\midrule

Baseline (VWAP buckets 2--7, cost=1) & 122 & 54.29 & 3.17 & 56.6\% \\
Exec: bucket 1 close (cost=1) & 62 & 6.45 & 0.17 & 45.2\% \\
Exec: full US VWAP (cost=1) & 122 & 41.39 & 2.35 & 56.6\% \\
Baseline (VWAP buckets 2--7, cost=2) & 122 & 4.29 & 0.25 & 43.4\% \\

\bottomrule
\end{tabular}

\normalsize

\subsection*{Baseline fixed-rule performance by expiry year}

\footnotesize

\begin{tabular}{lrrrr}
\toprule
Expiry yr & N & Net mean (USD/ct) & Net t & Win \\
\midrule

2008 & 3 & -205.53 & -0.94 & 33.3\% \\
2009 & 8 & 63.31 & 1.35 & 62.5\% \\
2010 & 10 & 154.50 & 1.44 & 60.0\% \\
2011 & 8 & 181.52 & 4.03 & 87.5\% \\
2012 & 8 & 1.66 & 0.03 & 37.5\% \\
2013 & 6 & 34.81 & 0.56 & 66.7\% \\
2014 & 6 & 54.49 & 0.54 & 66.7\% \\
2015 & 5 & -87.21 & -0.77 & 40.0\% \\
2016 & 8 & 12.10 & 0.51 & 37.5\% \\
2017 & 9 & 6.88 & 0.23 & 33.3\% \\
2018 & 9 & 124.21 & 4.89 & 100.0\% \\
2019 & 8 & 107.48 & 1.46 & 62.5\% \\
2020 & 6 & -33.67 & -0.86 & 33.3\% \\
2021 & 7 & -26.30 & -0.48 & 28.6\% \\
2022 & 5 & 188.10 & 1.54 & 80.0\% \\
2023 & 9 & 46.02 & 1.18 & 55.6\% \\
2024 & 7 & 74.01 & 1.63 & 57.1\% \\
Pooled & 122 & 54.29 & 3.17 & 56.6\% \\

\bottomrule
\end{tabular}

\normalsize


\end{document}
